
\documentclass[14pt,oneside,a4paper]{book}
\usepackage{polski}
\usepackage[utf8]{inputenc} 
\usepackage{lmodern}
\usepackage{indentfirst}
\usepackage{microtype}
\DisableLigatures{encoding = *, family = * }
\usepackage{fancyhdr}
\usepackage{pstricks,graphicx}
\usepackage{amssymb}
\usepackage{graphicx} % tutaj
\usepackage{grffile}
\graphicspath{{images/}}
\usepackage{wrapfig}
\usepackage{lipsum}
\def\bibname{Literatura}
\newcommand{\R}{\mathbb{R}}
\newcommand{\N}{\mathbb{N}}
\newcommand{\K}{\mathbb{K}}
\newcommand{\C}{\mathcal{C}}
\newcommand{\p}{\mathcal{P}}

\newcommand{\fal}{\mbox{{\Large $\forall\,$}}}
\newcommand{\ext}{\mbox{{\Large $\exists\,$}}}

\usepackage{theorem}
\theoremstyle{break}
\theorembodyfont{\it}
\newtheorem{twr}{Twierdzenie}[chapter]
\newtheorem{lem}{Lemat}[chapter]
\theorembodyfont{\rm}
\newtheorem{defi}{Definicja}[chapter]
\newtheorem{wni}{Wniosek}[chapter]
\newtheorem{prz}{Przykład}[chapter]
\newenvironment{dowod}{\par\vspace{0.1cm}\par{ \sc Dowód.}}{\hfill $\blacksquare$\par\vspace{0.4cm}\par}
% ----------ustawienia wymiarow strony
\usepackage{geometry}

\newgeometry{tmargin=2.5cm, bmargin=2.5cm, headheight=14.5pt, inner=3cm, outer=2.5cm} 

\linespread{1.1} %-zmiana interlinii


\usepackage{amsmath}


\pagestyle{fancy} 

\fancyhead[C]{} 
\fancyfoot[C]{\thepage}
\fancyhead[L]{\scriptsize\leftmark}
\fancyhead[R]{\scriptsize\rightmark}

\renewcommand{\chaptermark}[1]{%
\markboth{\MakeUppercase{%
\chaptername}\ \thechapter.%
\ #1}{}}

\renewcommand{\sectionmark}[1]{\markright{\thesection.\ #1}}


\usepackage{lipsum} 
\begin{document}
\thispagestyle{empty}
\begin{center}{\sc \Large
Wyższa Szkoła Bankowa\\
Wydział Zamiejscowy w Chorzowie}\par\vspace{0.2cm}\par
{\large
Zadanie 4 Poligrafia
}
\end{center}
\vspace{5cm}
\begin{center}
{\large
Marcin Krogulewski
}\par\vspace{0.2cm}\par
{\LARGE
Moje Ulubione Sporty
}
\end{center}
\vspace{4cm}
\begin{flushright}


\end{flushright}
\vfill
\begin{center}
27.02.2022
\end{center}

\newpage

\thispagestyle{empty} \setcounter{page}{0} \tableofcontents

\chapter*{Sporty Walki}

 \addcontentsline{toc}{chapter}{Mixed Martial Arts (MMA)}

Mieszane sztuki walki (ang. mixed martial arts, MMA) – dyscyplina sportowa, w której zawodnicy sztuk i sportów walki walczą przy dużym zakresie dozwolonych technik (w zasadzie dopuszcza się wszystkie techniki dozwolone w innych sportach walki bez broni). Mieszane sztuki walki wyrażają jeden ze współczesnych kierunków rozwoju sztuk walki – zapewnienie widowiska sportowego, w którym walka toczy się przy jak najmniejszych ograniczeniach, ale jednoczesnym zminimalizowaniu ryzyka śmierci i poważnych, trwałych obrażeń ciała lub kalectwa.


\begin{tabular}{|lc|r|}

\hline
Organizacja & Region\\
\hline
Ultimate Fighting Championship (UFC) &  Zasięg Globalny\\
Bellator MMA &  Zasięg Globalny\\
Absolute Championship Akhman (ACA) &  Zasięg Globalny\\
Extreeme Fighting Championship (EFC) &  Afryka\\
World Series of Fighting &  Ameryka Północna\\
Jungle Fight &  Ameryka Południowa\\
One Championship &  Azja\\
Cage Warriors &  Europa\\
Konfrontacja Sztuk Walki (KSW) &  Europa\\
M-1 Global &  Europa\\
\hline
\end{tabular}

Rywalizacja toczy się zarówno w stójce, jak i parterze. W typowych walkach MMA dozwolone są rzuty, ciosy pięściami, kopnięcia, dźwignie, duszenia. Zabronione są natomiast techniki stwarzające znaczne niebezpieczeństwo dla zdrowia zawodników. W regulaminach większości organizacji czy zawodów zakazane jest zwykle: gryzienie, zahaczanie (wkładania palców w otwory fizjologiczne, np. usta czy nos), atakowanie genitaliów, oczu i krtani, uderzanie głową, uderzanie w kręgosłup, stosowanie dźwigni na małe stawy, czyli palce. Często organizatorzy wprowadzają dodatkowe ograniczenia, zabraniając np. ciosów łokciami, dźwigni na kręgosłup, dźwigni skrętowych na kolana, wykonywania rzutów skutkujących upadkiem rywala na głowę albo kopnięć w parterze. Jeszcze bardziej restrykcyjne reguły obowiązywały dawniej w niektórych japońskich organizacjach (np. Pancrase czy RINGS), gdzie zabronione były uderzania w parterze, czy też uderzania zaciśniętą pięścią.O wyniku walki decyduje nokaut (również techniczny), poddanie się oraz, jeśli walka nie zakończyła się przed czasem, decyzja sędziów. W zależności od regulaminu danych zawodów niekiedy dopuszczalnym rozstrzygnięciem może być też remis.

\section {Typy decyzji}

jednogłośna – trzech sędziów wskazuje zwycięstwo punktowe tego samego zawodnika;
niejednogłośna – dwóch sędziów wskazuje zwycięstwo punktowe jednego zawodnika, a trzeci z sędziów ocenia walkę na korzyść jego rywala;
większościowa - dwóch sędziów wskazuje zwycięstwo punktowe jednego zawodnika, a trzeci z sędziów ocenia walkę jako remisową.
Walki odbywają się na zwykłym ringu bokserskim lub na ringach różnych kształtów otoczonych siatką, która zapobiega wypadaniu zawodników poza miejsce walki (potocznie zwanymi „klatkami”). Ośmiokątny ring używany przez UFC jest określany „oktagonem” (The Octagon – nazwa jest zastrzeżonym znakiem towarowym).
Zawodnicy zawsze muszą posiadać ochraniacze na zęby oraz rękawice (zwykle cienkie, umożliwiające chwytanie). Niekiedy dopuszcza się również stosowanie nagolenników lub ochraniaczy na kolana.

\section {Boks}

 Boks, pięściarstwo – sport walki, w którym dwóch zawodników walczy ze sobą jedynie przy użyciu pięści.

\section {Ciosy bokserskie}

Ciosy bokserskie stanowią podstawowy element techniki w walce bokserskiej, ciosy bokserskie są prawidłowe jeżeli zadawane są przednią, wypchaną częścią zamkniętej rękawicy (często oznaczoną na biało) w przednie i boczne części głowy (do linii uszu) oraz powyżej pasa z przodu i z boku do linii ramion opuszczonych luźno wzdłuż tułowia, rozróżnia się: ciosy sierpowe, ciosy proste, ciosy z dołu. W zależności od tego, która ręka zadaje ciosy rozróżnia się ciosy prawe i lewe; w zależności od celu, w który wymierzony jest cios, dzieli się je na ciosy na górę (w głowę) i ciosy na dół (w tułów); w zależności od ich zasięgu wyróżnia się ciosy krótkie i długie.
Ciosy proste – ciosy bokserskie najczęściej stosowane w walce, charakteryzujące się duża szybkością i skutecznością, stosowane są w ataku na dystans i w półdystansie oraz jako kontrciosy w obronie, rozróżnia się cztery podstawowe ciosy proste: lewy prosty na górę (w głowę), lewy prosty na dół (w tułów), prawy prosty na górę (w głowę) oraz prawy prosty na dół (w tułów). W angielskiej terminologii bokserskiej cios wykonywany tą ręką, po której stronie znajduje się wysunięta do przodu noga, określa się jako jab, zaś przeciwną ręką jako cross.
Ciosy sierpowe – ciosy boczne, które dochodzą do przeciwnika z boku, trafiając w boczne części głowy oraz tułów, stosowane najczęściej w półdystanse. Ciosy sierpowe charakteryzują się silną pracą skrętną tułowia z przeniesieniem ciężaru ciała w kierunku ciosu przy współudziale pracy nóg, bioder i barków. Podstawowymi ciosami sierpowymi są lewy sierpowy, prawy sierpowy, lewy sierpowy wydłużony oraz prawy sierpowy wydłużony. W angielskiej terminologii określane jako hook.
Ciosy z dołu (podbródkowe, haki) – ciosy bokserskie zadawane są ręką ugiętą w łokciu, stosowane w półdystansie i w zwarciu, w ataku i obronie, podstawowymi ciosami z dołu są prawy z dołu i lewy z dołu. W angielskiej terminologii określane jako uppercut.

\section {Pozycja Bokserska}

\begin{wrapfigure} {r}{40mm}

  \includegraphics {poz.jpg}

\end{wrapfigure}

\section {Pozycje Obronne}

Pozycje Obronne:

Odskok – element techniki stosowany w obronie, w którym bokser wychodzi poza zasięg ciosów przeciwnika przez odbicie z obu nóg. Odskok jest skuteczną obroną bierną przed każdym rodzajem ciosów.
Odchylenie – element techniki stosowany w obronie polegający na cofnięciu tułowia poza zasięg ciosów przeciwnika, przy czym nogi i biodra ustawione w pozycji bokserskiej pozostają nieruchome, odchylenie stosowane jest w obronie przed ciosami prostymi na górę oraz ciosami sierpowymi.
Garda – element techniki stosowany w walce bokserskiej jako obrona bierna; polega na takim ustawieniu rękawic i przedramion, by chroniły boksera przed ciosami przeciwnika.
Zakrok – cofnięcie nogi w celu uniknięcia ciosu i następnie wyprowadzenia kontry

\section {Wyposażenie boksera}

Sprzęt ochronny:
Kask ochronny – sprzęt ochronny boksera zabezpieczający części głowy (łuki brwiowe, uszy) przed urazami; najczęściej wykonywane są ze skóry, wewnątrz wypełnione gąbką. Od 1 czerwca 2013 roku Międzynarodowa Federacja Boksu (AIBA) zdecydowała, że pięściarze amatorzy będą rywalizować we wszystkich zawodach międzynarodowych bez kasków[3].
Ochraniacze podbrzusza – inaczej suspensoria, jest to podpaska mosznowa wykonana z tkaniny elastycznej w kształcie woreczka przymocowanego tasiemkami do paska opasującego biodra, ochraniacz stosowany jest obowiązkowo.
Ochraniacze zębów i warg – tzw. gumowa szczęka, część wyposażenia osobistego boksera, wykonane są z miękkiego tworzywa (najczęściej kauczuku) w kształcie półokrągłej rynienki, którą można dopasować do górnej szczęki, ochraniacz ten pomaga zabezpieczyć przed utratą zębów (lecz nie chroni całkowicie - znane były przypadki wybicia zębów podczas walki, pomimo używania ochraniacza, np. Mike Tyson) i rozbiciem warg, gdyż siła ciosu jest amortyzowana przez miękką i gładką powierzchnię oraz zabezpiecza przed wybiciem żuchwy z zawiasów, wymuszając konieczność zaciskania szczęki; ochraniacz stosowany jest obowiązkowo.
Pozostałe:
Buty – buty sznurowane sięgające powyżej kostki, wykonane z miękkiej skóry, z cienką podeszwą gumową bez obcasa; nie powinny posiadać żadnych usztywnień i części metalowych.
Spodenki – wykonane z materiału np. płótna, popeliny lub grubego jedwabiu, sięgają do połowy uda i posiadają pasek szerokości około siedmiu centymetrów, wykonany z trzech lub czterech gum oddzielnie wszytych w materiał innego koloru niż spodenki.
Rękawice bokserskie – podstawowy sprzęt boksera, rękawice wykonane są ze skóry, wewnątrz posiadają wyściółkę z włosia końskiego lub spienionego poliuretanu, która stanowi minimum 50 procent masy całych rękawic; zastosowanie wyściółki ma na celu amortyzowanie siły ciosu i dlatego musi być ona umocowana na stałe, nie może się przesuwać ani zmieniać kształtu. Od wewnątrz rękawice wykończone są płótnem, powinny być możliwie nowe, czyste i jednakowe dla obu zawodników.
Owijka (tzw. bandaże lub taśmy) – specjalne bandaże, zakładane na dłonie pod rękawice bokserskie, ich zadaniem jest wchłanianie potu oraz stabilizacja nadgarstka i kości śródręcza, co pomaga zapobiegać kontuzjom.


\pagestyle{fancy}


\chapter {Kick-Boxingi}
Kick-boxing, także kick boxing – dyscyplina sportowa (sport walki), w której walczy się stosując zarówno bokserskie ciosy pięścią, jak i kopnięcia. Sport rozwijający w sposób holistyczny umiejętności fizyczne takie jak: siła, szybkość, wytrzymałość, gibkość, poczucie rytmu. Dodatkowo rozwijający cechy psychiczne m.in.: panowanie nad stresem, poczucie własnej wartości, czy pewność siebie.

	Rywalizacja sportowa odbywa się według licznych regulaminów i formuł, wśród których można wyróżnić między innymi:
full contact: dozwolone są wszystkie techniki nożne i bokserskie powyżej pasa, z minimalną liczbą 8 kopnięć podczas rundy, jeśli zawodnik nie zada 8 kopnięć traci 1 punkt; nie wolno kopać kolanem ani uderzać łokciem; walka odbywa się na ringu bokserskim, standardowo trwa 3x2 min.
low kick: dozwolone są wszystkie techniki bokserskie oraz kopnięcia do wysokości głowy, nie ma minimalnej liczby kopnięć na minutę, nie wolno kopać kolanem ani uderzać łokciem; walka odbywa się na ringu, standardowo trwa 3x2 min.
K-1 Style: formuła wprowadzona i spopularyzowana przez japońską organizację K-1, dozwolone są wszelkie techniki bokserskie, obrotowe uderzenia pięścią, kopnięcia oraz ciosy kolanami, bez względu na wysokość; nie wolno uderzać łokciem; walka odbywa się na ringu, standardowo trwa 3x3 lub 5x3 min
pointfighting (dawniej semi contact) : formuła nastawiona głównie na szybkość zawodników; walka jest przerywana i punktowana po każdym czystym trafieniu przeciwnika; ograniczona siła uderzeń; walka odbywa się na tatami, wyjątkowo na ringu, standardowo trwa 3x2 min.
light contact: forma walki ciągłej w której zawodnicy muszą wykazać się umiejętnością walki technicznej; walka odbywa się na ringu, parkiecie lub tatami, standardowo trwa 3x2 min.
kick light: forma walki ciągłej z kopnięciami od uda w górę oraz ograniczoną siłą uderzeń; walka odbywa się na ringu, parkiecie lub tatami.
Techniki kopnięć:
	Podstawowe:
- kopnięcie frontalne (ang. front kick)
- kopnięcie okrężne (ang. roundhouse kick, round kick lub turning kick)
- kopnięcie boczne (ang. side kick)
- kopnięcie opadające (ang. axe kick)
- kopnięcie haczące (ang. heel kick)
- kopnięcie ścinające
	Obrotowe:
- kopnięcie boczne z obrotem
- kopnięcie ściągające (opadające) z obrotem
- kopnięcie zahaczające (hakowe) z obrotem
- kopnięcie okrężne z obrotem
	Z wyskoku:
- kopnięcie frontalne z wyskoku
- kopnięcie okrężne z wyskoku
- kopnięcie boczne z wyskoku
- kopnięcie zahaczające z wyskoku
	Z wyskokiem i obrotem:
- boczne z wyskokiem i obrotem
- zahaczające z wyskokiem i obrotem

\section {Stopnie Szkoleniowe}

	Uczniowskie:
10. biały pas
9. biały pas z żółtym zakończeniem pasa
8. żółty pas
7. żółty pas z pomarańczowym zakończeniem pasa
6. pomarańczowy pas
5. zielony pas
4. niebieski pas
3. brązowy pas z niebieskim zakończeniem pasa
2. brązowy pas
1. brązowy pas z czarnym zakończeniem pasa
	Mistrzowskie:
1. czarny pas


\chapter {Piłka Nożna}

Piłka nożna (futbol, ang. football, association football, soccer) – gra zespołowa, w której dwie drużyny starają się zdobyć w określonym czasie jak najwięcej punktów poprzez wbicie piłki do bramki; najpopularniejsza dyscyplina sportowa z około 4 miliardami fanów na całym świecie.

\section {Zasady}

Mecze rozgrywane są na polu gry wyznaczonym w postaci prostokąta o szerokości od 45 do 90 m i długości od 90 do 120 m (przy jednoczesnym zastrzeżeniu, że boisko nie może być kwadratem, dla meczów międzynarodowych od marca 2008 FIFA ustanowiła wymiary boisk 105x68). Dwie krótsze linie nazywają się liniami końcowymi przy czym odcinek między słupkami – linią bramkową, natomiast dwie dłuższe – liniami bocznymi. Po przeciwległych stronach pola gry, na środku linii bramkowych, ustawione są bramki o szerokości między wewnętrznymi krawędziami słupków 7,32 m (8 yd) i wysokości dolnej krawędzi poprzeczki od podłoża 2,44 m (8 ft). Zawody są rozgrywane piłką, która powinna mieć obwód nie mniejszy niż 68 cm i nie większy niż 70 cm, a jej masa powinna wynosić od 410 do 450 g (16 oz). Przy rozpoczęciu zawodów ciśnienie powietrza we wnętrzu piłki musi wynosić od 0,6 do 1,1 atmosfery.
Pełny skład drużyny liczy 11 zawodników, w tym bramkarz.  Drużyna może również wyznaczyć maksymalnie do 7 zawodników rezerwowych (w rozgrywkach międzynarodowych rangi mistrzowskiej do 12). Personalia wszystkich zawodników muszą być wpisane do sprawozdania sędziowskiego, które musi być dostarczone do sędziego przed rozpoczęciem zawodów. W trakcie spotkania drużyna może dokonywać wymian zawodników, których liczba zależy od regulaminu danych rozgrywek. Zawodnik wymieniony zazwyczaj nie może znaleźć się ponownie na placu gry.
Zawodnicy jednej drużyny (poza bramkarzem) noszą w trakcie zawodów taki sam ubiór, który odróżnia ich od zawodników drużyny przeciwnej. Obowiązkowy ubiór zawodnika składa się z koszulki, spodenek, getrów piłkarskich oraz butów. Każdy zawodnik musi posiadać również ochraniacze goleni, które muszą być całkowicie przykryte getrami. Koszulka musi posiadać rękawy i stanowić odrębną część ubioru od spodenek (niedozwolone jest używanie przez drużyny strojów jednoczęściowych). Koszulka musi mieć na plecach numer w kolorze kontrastującym z kolorem koszulki, natomiast inne elementy dekoracyjne (emblematy, loga klubów i sponsorów, nazwiska piłkarzy) są regulowane odrębnymi zapisami w regulaminach danych rozgrywek.
Piłkę w czasie gry można uderzać głową, nogą, przyjmować na klatkę piersiową itp., nie wolno jedynie rozmyślnie zagrywać jej rękami. Zakaz ten nie dotyczy bramkarza zagrywającego piłkę znajdującą się w obrębie własnego pola karnego, poza sytuacją rozmyślnego podania od pasa w dół do bramkarza przez współpartnera lub podania piłki do bramkarza z wrzutu. Rozmyślne dotknięcie piłki ręką jest karane rzutem wolnym bezpośrednim dla drużyny przeciwnej (lub rzutem karnym, jeśli piłkę ręką zagrał zawodnik z pola we własnym polu karnym). W zależności od sytuacji, sędzia może również ukarać zawodnika, który przewinił karą indywidualną w postaci żółtej lub czerwonej kartki.

\section {Wykroczenia i Kary}

Przepisy gry w piłkę nożną definiują szereg przewinień, za które drużyna zawodnika może być ukarana rzutem wolnym lub rzutem karnym. Rzuty wolne dzielą się na rzuty wolne bezpośrednie i pośrednie. Rzut wolny bezpośredni może być przyznany przeciwko drużynie, której zawodnik dopuszcza się w czasie gry i na polu gry jednego z następujących przewinień: kopie lub usiłuje kopnąć przeciwnika, podstawia bądź próbuje podstawić nogę przeciwnikowi, skacze na przeciwnika, nieprawidłowo atakuje przeciwnika ciałem, uderza lub usiłuje uderzyć przeciwnika, popycha przeciwnika, atakuje przeciwnika nogami, trzyma przeciwnika, pluje na przeciwnika, atakuje sędziego lub rozmyślnie dotyka piłkę ręką. Jeżeli któreś z tych przewinień zostaje dokonane w obrębie pola karnego drużyny zawodnika – sędzia przyznaje drużynie przeciwnej rzut karny. Z rzutu wolnego bezpośredniego bramka może zostać zdobyta bezpośrednio, ale tylko na drużynie przeciwnej. W przypadku wykonania rzutu wolnego w kierunku własnej bramki (ale spoza własnego pola karnego), jeżeli piłka niedotknięta przez żadnego zawodnika przekroczy całym obwodem linię bramkową pomiędzy słupkami i pod poprzeczką, sędzia przyzna rzut rożny dla drużyny przeciwnej. Jeżeli w powyższej sytuacji rzut wolny bezpośredni będzie wykonywany z własnego pola karnego – sędzia nakaże powtórzyć jego wykonanie – wynika to z faktu, że piłka nie jest w grze, dopóki nie zostanie kopnięta bezpośrednio poza własne pole karne w obrębie pola gry. Rzut wolny pośredni może zostać przyznany jeżeli: bramkarz przez czas dłuższy niż 6 sekund kontroluje piłkę we własnych rękach i nie pozbędzie się jej, bramkarz dotknie piłki rękami (we własnym polu karnym) po raz drugi po tym, jak wypuścił ją z rąk, a nie została dotknięta przez innego zawodnika, bramkarz rozmyślnie dotknie piłki ręką we własnym polu karnym po rozmyślnym podaniu jej nogą (poniżej kolana) od współpartnera, bramkarz rozmyślnie dotknie piłki rękoma we własnym polu karnym po otrzymaniu jej bezpośrednio z wrzutu od współpartnera, zawodnik gra w sposób niebezpieczny (np. atakuje piłkę nogą wyprostowaną do przodu, atakuje piłkę nogą podniesioną powyżej biodra przeciwnika będącego w jego zasięgu, atakuje piłkę głową poniżej biodra przeciwnika będącego w jego zasięgu itp), zawodnik przeszkadza bramkarzowi drużyny przeciwnej w zwolnieniu piłki z rąk, zawodnik popełnia inne przewinienie, niewymienione w przepisach gry w piłkę nożną, z którego powodu sędzia przerwał grę w celu udzielenia zawodnikowi kary indywidualnej. Rzut wolny pośredni jest również przyznawany drużynie przeciwnej, jeżeli zawodnik w czasie gry i na polu gry kopie lub usiłuje kopnąć współpartnera, uderza lub usiłuje uderzyć współpartnera, wchodzi na pole gry bez zgody sędziego i wpływa na grę lub jest winny niesportowego zachowania. Z rzutu wolnego pośredniego nie można zdobyć bramki bezpośrednim strzałem. Jeżeli piłka po wykonaniu rzutu wolnego pośredniego wpadnie bezpośrednio do bramki przeciwnika – sędzia przyzna rzut od bramki, natomiast jeżeli wpadnie do bramki wykonawcy – zasada jest identyczna jak przy wykonywaniu rzutu wolnego bezpośredniego.
Sędzia nie ma obowiązku natychmiastowego przerywania gry w przypadku popełnienia przewinienia – ma prawo do zastosowania przywileju korzyści – w takiej sytuacji jeżeli korzyść zostanie zrealizowana – sędzia nie wraca do dyktowania rzutu wolnego, a kontynuuje grę. W przypadku, kiedy dwóch zawodników przeciwnych drużyn popełnia identyczne przewinienie w tym samym czasie, w czasie gry, lub kiedy zawodnik popełnia przewinienie w czasie gry, ale poza polem gry – sędzia przerywa grę, która będzie wznowiona rzutem sędziowskim.
Sędzia niezależnie od kar zespołowych, może również udzielić konkretnemu zawodnikowi lub zawodnikom kary indywidualnej w postaci napomnienia (żółta kartka) lub wykluczenia z gry (czerwona kartka, która jest też efektem 2 żółtych kartek w jednym meczu). Karami indywidualnymi mogą zostać ukarani również zawodnicy rezerwowi oraz zawodnicy wymienieni, nie mogą zostać jednak ukarane osoby towarzyszące drużynom, uprawnione do przebywania w strefie technicznej – sędzia ma prawo pozbawić tych osób przywileju przebywania w strefie technicznej, nie może jednak pokazać kartki. Karę napomnienia otrzymuje zawodnik, który: jest winny niesportowego zachowania, słownie lub czynnie okazuje niezadowolenie, uporczywie narusza przepisy gry, opóźnia wznowienie gry, nie zachowuje wymaganej odległości podczas wykonywania rzutu wolnego, rzutu z rogu lub wrzutu, wchodzi lub powraca na pole gry bez zgody sędziego, rozmyślnie opuszcza pole gry bez zgody sędziego. Karę napomnienia sędzia udziela również zawodnikowi, który przerywa w niedozwolony sposób korzystnie rozwijającą się akcję przeciwników (faul taktyczny). Kary wykluczenia z gry sędzia udziela zawodnikowi który: popełnia poważny, rażący (brutalny) faul, zachowuje się gwałtownie, agresywnie, pluje na przeciwnika lub inną osobę, pozbawia drużynę przeciwną bramki lub realnej szansy na zdobycie bramki, używa ordynarnego, obelżywego języka i/lub gestów, otrzymuje drugie napomnienie w tych samych zawodach. Kary indywidualnej sędzia udziela pokazując kartkę w odpowiednim kolorze w ręce uniesionej do góry.

\section {Stałe Fragmenty Gry}

Jeśli piłka po uderzeniu, podaniu lub odbiciu przekroczy całym obwodem linię boczną, sędzia wskazuje wrzut. Wznowienie gry następuje przez wrzut piłki rękoma, zza i znad głowy bez odrywania nóg od podłoża z miejsca, w którym piłka opuściła pole gry, przez zawodnika drużyny przeciwnej. Gdy piłka opuści boisko poprzez linię bramkową (ale poza bramką), grę rozpoczyna się rzutem z rogu lub rzutem od bramki, w zależności od tego, który zawodnik ostatni dotknął piłki. Rzut rożny wykonywany jest przez drużynę atakującą, jeżeli ostatnim zawodnikiem, który dotknął piłki, jest zawodnik broniący; w przeciwnej sytuacji wykonywany jest rzut od bramki. Rzut rożny wykonywany jest z pola rożnego i obowiązują przy jego wykonaniu takie same zasady jak przy wykonywaniu rzutu wolnego bezpośredniego. Rzut od bramki wykonywany jest z dowolnego miejsca pola bramkowego wykonawcy, przy czym piłka po zagraniu musi niedotknięta przez nikogo opuścić pole karne w obrębie boiska.
Po faulu w obrębie pola karnego sędzia dyktuje rzut karny. Rzut karny wykonywany jest z punktu karnego, który znajduje się w odległości 11 m od linii bramkowej oraz w równej odległości od słupków bramkowych. W czasie wykonywania rzutu karnego w polu karnym do momentu wprowadzenia piłki do gry może znajdować się tylko zawodnik wykonujący rzut karny oraz bramkarz, któremu nie wolno opuszczać linii bramkowej (ani skakać na niej obiema nogami). Rzut karny może być również rozegrany pomiędzy dwoma zawodnikami z drużyny, jednak po podaniu piłka musi potoczyć się do przodu, w kierunku bramki przeciwnika, a zawodnik, do którego podanie jest adresowane, do momentu zagrania piłki musi znajdować się poza polem karnym oraz w odległości 9 m 15 cm od punktu karnego i w odległości większej niż 11 m od linii bramkowej. Dozwolone jest dobijanie strzałów przez wszystkich zawodników. Strzelec rzutu karnego może dobijać, ale tylko jeżeli piłka zostanie rozegrana z innym zawodnikiem (najczęściej bramkarzem drużyny przeciwnej lub obrońcą). Jeżeli piłka po strzale zostanie odbita od słupka i nie dotknie jej żaden inny zawodnik, strzelec nie może ponownie zagrać piłki (a jeżeli dotknie – zostanie podyktowany rzut wolny pośredni dla drużyny przeciwnej). Wykonanie rzutu karnego powinno być płynne, niedozwolone jest markowanie strzału. W przypadku nieprawidłowości sędzia powtarza wykonanie rzutu karnego.

\section {Pozycja Spalona}

Zawodnik drużyny atakującej znajduje się na pozycji spalonej, gdy jest na połowie przeciwnika i w momencie podania do niego piłki ma przed sobą mniej niż dwóch graczy drużyny przeciwnej.
 
W szczególności wynika z tego, że na spalonym nie jest piłkarz, który jest za linią obrony, ale nie przekroczył linii połowy boiska. Spalony nie obowiązuje również podczas wyrzutu z autu oraz w sytuacji, gdy potencjalnie „spalony” zawodnik w momencie podania kieruje się spokojnym krokiem w stronę własnej połowy (tzn. ewidentnie rezygnuje z podejmowania akcji ofensywnej). W przypadku spalonego sędzia najczęściej kieruje się wskazaniem bocznych arbitrów i – w przypadku odnotowania pozycji spalonej – dyktuje rzut wolny pośredni dla drużyny przeciwnej.

\section {Przywilej Korzyści}

Zasadą, którą często kierują się sędziowie, jest zasada korzyści. Oznacza ona rezygnację z odgwizdania, jeśli drużyna pokrzywdzona jest w korzystnej sytuacji. Przykładowo: jeśli podczas wykonywania rzutu wolnego mur wbiegnie w strefę 9,15 m od piłki, a mimo to padnie gol, to sędzia nie nakaże powtórzenia rzutu wolnego. Podobnie będzie, gdy nastąpi faul na kartkę, lecz sędzia zdecyduje pokazać ją po zakończeniu akcji, gdyż pokrzywdzona drużyna pomimo faulu jest w posiadaniu piłki; lub gdy bramkarz sfauluje napastnika, a ten strzeli bramkę – zostanie ona uznana; gdy obrońca dotknie piłki ręką, a mimo to drużyna atakująca po chwili strzeli bramkę – również zostaje ona uznana.

\section {Czas Trwania Rozgrywki}

Czas gry wynosi zazwyczaj 90 minut (dwie połowy, każda po 45 minut). Przerwa między połowami wynosi 15 minut. Sędzia może przedłużyć każdą połowę meczu stosownie do przerw w grze. Po upływie doliczonego czasu gry zwycięzcą jest ta drużyna, która zdobyła więcej bramek. W przypadku rozgrywania meczów systemem ligowym, za wygraną zwycięzca zdobywa 3 punkty (w niektórych ligach 2 punkty), przegrany nie zdobywa żadnego. Remis natomiast obu drużynom daje po jednym punkcie. W systemie ligowym drużyny grają ze sobą każdy z każdym po dwa razy, jednak np. w lidze austriackiej i szkockiej drużyny rozgrywają między sobą po 4 pojedynki.
W przypadku, gdy mecze rozgrywane są turniejowym systemem dwumeczów, pierwszy mecz zawsze kończy się po drugiej połowie. Zwycięzcą dwumeczu zostaje ta drużyna, która w dwóch meczach zdobędzie więcej goli. Jeśli okaże się, że obie drużyny zdobyły ich tyle samo, odbywa się dogrywka.
	Jeśli w przepisowym czasie gry drużyny osiągną wynik remisowy, a konieczne jest wyłonienie zwycięzcy (co ma miejsce w przypadku spotkań rozgrywanych systemem turniejowym bez rewanżów), ma wówczas miejsce trzydziestominutowa dogrywka – 2 połowy po 15 minut. W większości rozgrywek na szczeblu międzypaństwowym obowiązywała do niedawna zasada „złotej bramki” (golden goal), polegająca na tym, że po strzeleniu bramki w dogrywce mecz się kończy, a zwycięzcą zostaje drużyna, którą ją zdobyła. Później na jej miejsce wprowadzono zasadę „srebrnej bramki” (silver goal), polegającej na tym, że zwycięzcą zostaje drużyna, która prowadzi po pierwszej połowie dogrywki. Obecnie odchodzi się od tych zasad na rzecz pełnowymiarowej dogrywki.

\section {Rzuty Karne}

Jeśli dogrywka nie przyniesie rozstrzygnięcia, wówczas zawodnicy obu drużyn wykonują serię rzutów karnych (po 5 rzutów karnych, każda dla każdej drużyny), a gdy i po nich nie ma zwycięzcy, wykonuje się na przemian po jednym rzucie karnym aż do osiągnięcia zwycięstwa. W serii rzutów karnych piłkarze na przemian wykonują rzuty karne z przepisowej odległości. Żaden gracz nie może oddać więcej niż jednego strzału, chyba że wszyscy pozostali gracze pola oddali już strzał w serii rzutów karnych. Gola uznaje się, gdy piłka wpadnie do bramki bezpośrednio, po odbiciu od słupka, poprzeczki lub od bramkarza. Nie uznaje się tzw. „dobitek”.
W niektórych systemach rozgrywek, jeśli dogrywka nie wyłoniła zwycięzcy, nie przewidywano serii rzutów karnych lub ograniczano ich liczbę. W takich przypadkach, jeśli wciąż nie było możliwe wskazanie zwycięzcy – sędzia rozstrzygał wynik wykonując rzut monetą.

\chapter {Siatkówka}

Piłka siatkowa, siatkówka – sport drużynowy, w którym biorą udział dwa zespoły po 6 zawodników w każdym (rozgrywający, atakujący, dwóch środkowych i dwóch przyjmujących, libero). Na boisku przebywa jednak tylko sześciu zawodników, libero zmienia się ze środkowym będącym w linii obrony, gdy drużyna przyjmuje zagrywkę.
Zasady
	Siatkówka polega na odbijaniu piłki dowolną częścią ciała (najczęściej rękoma) tak, aby przeleciała nad siatką i dotknęła połowy boiska należącej do przeciwnika. Każdy zespół może wykonać trzy odbicia – odbiór, wystawa i atak - każde następne jest błędem. Punkty zdobywa się na wiele sposobów, na przykład, gdy piłka dotknie połowy boiska przeciwnika lub gdy upadnie ona poza boiskiem, a rywal dotknął jej jako ostatni. Gra toczy się do momentu, gdy jedna z drużyn wygra 3 sety (w grupach młodzieżowych 2 – młodziczki i młodzicy) - w każdym z nich rywalizacja toczy się do zdobycia 25 punktów przy przewadze co najmniej dwóch punktów. Przy wyniku 24:24 gra toczy się "na przewagi", dopóki któraś z drużyn osiągnie dwupunktową przewagę nad drugą. Wyjątek stanowi piąty set, nazywany "setem decydującym", który jest rozgrywany, jeśli stan meczu po 4 setach wynosi 2:2. W tym secie zawodnicy grają do 15 punktów, również przy przewadze co najmniej dwóch punktów.
Zasady gry są ustalone przez Międzynarodową Federację Piłki Siatkowej i zawarte są w przepisach gry w piłkę siatkową. Modyfikacji przepisów dokonuje FIVB, zwykle przy okazji ważniejszych imprez, tj. mistrzostw świata czy igrzysk olimpijskich. Przepisy ewoluowały na przestrzeni lat. Największe w ostatnich latach zmiany, opisane poniżej, zostały wprowadzone w 1998 roku, podczas mistrzostw świata w Japonii. Wcześniej set kończył się po zdobyciu 15 punktów, a drużyna zdobywała je tylko przy własnej zagrywce, nie istniała także pozycja Libero. Ostatnia duża nowelizacja przepisów miała miejsce w 2008 roku, wtedy w zawodach organizowanych przez FIVB weszła ona w życie 1 stycznia 2009 roku, a w rozgrywkach w Polsce zaś od sezonu 2009/2010. Zmiany te dotyczyły: stosunku do kontaktu zawodnika z siatką – stał się on bardziej liberalny; w uproszczeniu − błędem było tylko dotknięcie górnej 7-centymetrowej taśmy, inny kontakt nie był uważany za błąd, chyba że wpływał na grę, oraz przyśpieszona została procedura zmian zawodników. Zgodnie z tymi przepisami może być do dwóch libero, a składy drużyn mogą liczyć maksymalnie 12 zawodników wpisanych do protokołu. W sezonie 2015/2016 przywrócono przepis zabraniający dotykania siatki w jakimkolwiek miejscu, a kontakt zawodnika z siatką jest uznawany za błąd i powoduje stratę punktu.

\section {Punkt, Set, Mecz}

Celem gry jest przebicie piłki nad siatką tak, by upadła na boisku drużyny przeciwnej lub zmuszenie rywali do popełnienia błędu (np. odbicia piłki w aut). Piłkę można odbijać (nie może być złapana lub rzucana) dowolną częścią ciała. Jeden i ten sam zawodnik nie może odbić piłki dwa razy z rzędu (z wyjątkiem pierwszego odbicia następującego po bloku, oraz pierwszej piłki przy odbiorze ataku – warunek: kilka odbić musi być w jednej akcji), a drużyna może piłkę odbić maksymalnie 3 razy (nie licząc dotknięcia piłki przez blok), zanim przebije ją na stronę przeciwnika.
Piłka przebijana na stronę przeciwnika musi przelecieć nad siatką w przestrzeni ograniczonej: od dołu – górną krawędzią siatki, na bokach – przez antenki i ich umowne przedłużenie w górę, od góry przez sufit sali.
Punkt przyznaje się za każdą wygraną akcję – tj. wtedy, gdy piłka upadnie na boisko przeciwnika, kiedy przeciwny zespół popełni błąd lub zostanie ukarany karą (druga żółta kartka).
W przeciwieństwie do większości gier zespołowych, czas gry nie jest limitowany. Mecz toczy się aż jedna z drużyn wygra 3 sety (tak więc gra się do maksymalnie 5 setów). Piąty, decydujący set rozgrywany jest obecnie jako tie-break (do 15 punktów) i nosi nazwę seta decydującego. Drużyna wygrywa set, jeśli zdobędzie co najmniej 25 punktów i ma co najmniej 2 punkty przewagi nad przeciwnikiem.
FIVB podjęło decyzję, że podczas zawodów Ligi Europejskiej w 2013 roku przetestuje nowy system punktowania. W setach 1-4 kończącym punktem będzie punkt 21 (z zachowaną dwupunktową przewagą nad przeciwnikiem – jak do tej pory). W tych setach będzie tylko jedna przerwa techniczna po zdobyciu przez jeden z zespołów 12 punktów w secie. Set "decydujący" pozostaje bez zmian. Zmiany mają ujednolicić system punktowy. Taki sam obowiązuje w siatkówce plażowej.

\section {Zagrywka}

Zagrywka (lub też serwis) jest wprowadzeniem piłki do gry mające na celu zdobycie punktu lub utrudnienie jej przyjęcia przeciwnikowi. Zagrywkę wykonuje się zza linii końcowej boiska ze strefy zagrywki sposobem dolnym lub górnym. Piłka po zagrywce może dotknąć siatki pod warunkiem, że przeleci na stronę przeciwnika. Drużyna przeciwna nie może blokować zagrywki ani atakować piłki, kiedy ta znajduje się całkowicie powyżej siatki i nad polem ataku.
Piłka jest zagrywana przez prawego zawodnika linii obrony z pola zagrywki, znajdującego się za 9 metrem boiska. Pierwszy sędzia zezwala na wykonanie zagrywki po stwierdzeniu, że zawodnik zagrywający posiada piłkę i oba zespoły są przygotowane do gry.
Pierwsza zagrywka w pierwszym secie oraz w tie-breaku wykonywana jest przez zespół ustalony w losowaniu. W pozostałych setach pierwszą zagrywkę wykonuje zespół, który nie wykonywał jej jako pierwszy w secie poprzednim. Każdą inną zagrywkę w secie wykonuje zespół, który jako ostatni zdobył punkt.
Zawodnik zagrywający musi się trzymać poniższych reguł:
- Piłka musi być uderzona dłonią lub dowolną częścią ręki po podrzuceniu jej w górę lub opuszczeniu w dół z dłoni.
- Dozwolone jest tylko jedno podrzucenie lub opuszczenie piłki. Dozwolone jest także kozłowanie piłki oraz jej przekładanie z ręki do ręki.
- W momencie wykonywania zagrywki czy też w chwili odbicia przy zagrywce z wyskoku, zawodnik nie może dotknąć boiska (ani linii końcowej) oraz podłoża na zewnątrz pola zagrywki. Po uderzeniu piłki zawodnik może opaść poza pole zagrywki lub na boisko.
- Zawodnik musi wykonać zagrywkę w ciągu 8 sekund od gwizdka sędziego.
- Zagrywka wykonana przed gwizdkiem sędziego pierwszego jest unieważniona i musi być powtórzona.

Zawodnicy zespołu zagrywającego nie mogą przez stosowanie indywidualnej lub zbiorowej zasłony utrudniać przeciwnikowi obserwację zagrywającego oraz lot piłki.

\section {Wymiary Boiska}

Boisko do gry jest prostokątem o wymiarach 18 na 9 metrów ograniczonym dwiema liniami końcowymi i dwiema liniami bocznymi i otoczonym strefą wolną o szerokości co najmniej 3 m z każdej strony (na zawodach organizowanych przez FIVB wolna strefa musi mieć co najmniej: 8 m za liniami końcowymi i 5 m za bocznym boiska). Wszystkie linie końcowe i boczne wykreślone są wewnątrz boiska. Oś linii środkowej dzieli boisko na dwa równe pola o wymiarach 9 na 9 m każde. Na każdej stronie wyznaczona jest strefa ataku, ograniczona linią środkową, liniami bocznymi i linią ataku znajdującą się 3 m od osi linii środkowej i wpisaną w strefę ataku. Ponadto istnieje strefa zagrywki o szerokości 9 m i głębokości równej szerokości wolnej strefy.
Boisko przedzielone jest siatką, umieszczoną nad osią linii środkowej. Jej górna krawędź powinna znajdować się na wysokości 2,43 m dla mężczyzn i 2,24 m dla kobiet (dla młodzików 2,35 m i młodziczek 2,15 m[a]). Na dwóch końcach siatki (nad liniami bocznymi) wysokość siatki powinna być taka sama, jednak nie może być większa niż 2 cm ponad wysokość przepisową.
Do siatki mocowane są tzw. antenki, wystające 80 cm ponad taśmę górną. Ograniczają one strefę przejścia piłki nad siatką. Jeśli piłka dotknie antenki, traktowane jest to jako aut.

\section {Skład i Ustawienie}

Mecze siatkówki rozgrywane są przez dwa sześcioosobowe zespoły, a łącznie z rezerwowymi jeden zespół może składać się maksymalnie z 14 zawodników, trenera, dwóch asystentów trenera, lekarza i fizjoterapeuty. Podstawowymi pozycjami w siatkówce są: przyjmujący, atakujący, środkowy, rozgrywający i libero. Kapitanem zespołu może zostać jeden z zawodników. Poprzednio nie mógł nim być libero, co zostało zmienione na trzydziestym siódmym Światowym Kongresie FIVB; od 1 stycznia 2022 roku libero może zostać kapitanem. Jeśli wyznaczony kapitan znajduje się na ławce rezerwowych, wyznaczany jest "grający kapitan", który traci uprawnienia automatycznie z powrotem wyznaczonego kapitana na boisko i tylko oni (kapitan wyznaczony lub grający) mają prawo w imieniu drużyny poprosić sędziego pierwszego o wyjaśnienie zastosowanej interpretacji przepisów przez sędziów.
Przed rozpoczęciem każdego seta trener musi przekazać kartkę z ustawieniem początkowym swojego zespołu, co jest niezbędne do kontrolowania porządku rotacji zawodników w czasie gry. Rotacja zawodników następuje wtedy, kiedy zespół odbierający zagrywkę zdobywa prawo do wykonywania zagrywki, wtedy zawodnicy dokonują zmiany pozycji, przesuwając się o jedną zgodnie z ruchem wskazówek zegara (zawodnik z pozycji 2 przechodzi na pozycję 1, zawodnik z pozycji 1 przechodzi na pozycję 6 itd.).
W momencie uderzenia piłki przez zawodnika zagrywającego, każdy zespół musi znajdować się na własnej połowie boiska, ustawiony zgodnie z porządkiem rotacji (z wyjątkiem zawodnika zagrywającego) – trzej zawodnicy linii ataku wzdłuż siatki (zajmują miejsca: 4 – lewy ataku, 3 – środkowy ataku, 2 – prawy ataku) i trzej z linii obrony (na miejscach: 5 – lewy obrony, 6- środkowy obrony i 1 – prawy obrony). Każdy z zawodników linii obrony musi znajdować się dalej od siatki niż odpowiadający mu zawodnik linii ataku. Po wykonaniu zagrywki zawodnicy mogą zajmować dowolną pozycję na swoim polu gry. W przypadku, gdy popełniono błąd ustawienia bądź rotacji, a piłka została już uderzona przez zagrywającego, zespół przegrywa akcję.

\par\vspace{1cm}\par
\begin{flushright}

\end{flushright}


\end{document}
